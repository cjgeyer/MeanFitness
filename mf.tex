
\documentclass[11pt]{article}

\usepackage{amsmath}
\usepackage{indentfirst}
\usepackage{natbib}
\usepackage[colorlinks=true,linkcolor=blue,urlcolor=blue,citecolor=blue]
    {hyperref}
\usepackage{doi}

\begin{document}

\title{Estimating Population Mean Fitness and Change Thereof When
    the Population Distribution is Unknown}

\author{Charles J. Geyer \and Ruth G. Shaw}

\maketitle

\section{Estimates}

\subsection{Experiment}

We have an experiment in which (Darwinian) fitness (measured as lifetime
number of offspring per individual) was measured under wild conditions.
But the individuals in the experiment were not a random sample from
a population but rather pedigreed experimental crosses.  There are no data on
population allele frequencies, and even if there were, there would be
no good way to treat these experimental crosses as members of some
population.  Thus we simply assume some population distribution (of
pedigreed crosses).  All of our inferences will be sensitive to this
assumption, but we can check how sensitive (by trying different assumptions
or doing other mathematical sensitivity analysis).

The particular experiment at hand has five subexperiments laid out as follows.
\begin{center}
\begin{tabular}{ccc}
year one & year two & year three \\
\hline
parents & parents & parents \\
        & offspring & offspring
\end{tabular}
\end{center}
Here ``parents'' means individuals from the pedigreed crosses (seeds from
the same crosses were planted in each year).  For short, we will say these
individuals are from the same ``family'' if they are known full sibs
(have the same parents).
Here ``offspring'' means individuals whose dam was a ``parent'' from the
previous year (sire unknown because of open pollination).
For short, we will say these
individuals are from the same ``family'' as their dam in the previous year.

\subsection{Change in Mean Fitness}

If
\begin{itemize}
\item $p_i$ is the (assumed) population frequency of family $i$ (perhaps
    assumed to be equal for all $i$, perhaps not), and
\item $\hat{\mu}_i$ is (estimated) mean fitness for family $i$,
\end{itemize}
then
\begin{itemize}
\item $\sum_i p_i \hat{\mu}_i$ is (estimated) mean fitness for
that subexperiment, where the sum runs over all families.
\end{itemize}

We also need to know the (estimated) population frequency
\emph{after selection}, which is
$$
   \hat{q}_i = \frac{p_i \hat{\mu}_i}{\sum_j p_j \hat{\mu}_j}
$$
where the sum in the denominator runs over all families.
(This is standard in quantitative genetics, Lande-Arnold, breeder's equation,
and so forth, but does not account for recombination and other effects.)

Now we need to consider three subexperiments
\begin{itemize}
\item a ``parent'' subexperiment in one year (denoted by superscript prev),
\item an ``offspring'' subexperiment in the next year (denoted
    by superscript off), and
\item the ``parent'' subexperiment in the same year as the ``offspring
    experiment (denoted by superscript par).
\end{itemize}

The total change in mean fitness in one generation is
\begin{equation} \label{eq:total-change}
   \sum\nolimits_i \bigl(
   \hat{q}_i^\text{prev} \hat{\mu}_i^\text{off}
   -
   p_i^\text{prev} \hat{\mu}_i^\text{prev}
   \bigr)
\end{equation}

\subsection{Decomposition of Change in Mean Fitness}

We break this up as the sum of three differences in mean fitness.
\begin{subequations}
\begin{equation} \label{eq:partial-change-fftns}
   \sum\nolimits_i
   \bigl(\hat{q}_i^\text{prev} - p_i^\text{prev}\bigr) \hat{\mu}_i^\text{prev}
\end{equation}
is the change in mean fitness that Fisher's fundamental theorem of
natural selection tries to address (change in fitness due to natural
selection), i.e. this is the genetically based change in mean fitness within
the generation/yr in which selection takes place.
\begin{equation} \label{eq:partial-change-environmental}
    \sum\nolimits_i \hat{q}_i^\text{prev} \big(\hat{\mu}_i^\text{par} -
    \hat{\mu}_i^\text{prev}\bigr)
\end{equation}
is the change in mean fitness due to different environments in the
two years.
\begin{equation} \label{eq:partial-change-genetic-non-fftns}
    \sum\nolimits_i \hat{q}_i^\text{prev} \bigl(\hat{\mu}_i^\text{off} -
    \hat{\mu}_i^\text{par}\bigr)
\end{equation}
is the change in mean fitness due to differences in genetics of the
parent and offspring subjects not due to natural selection
(something not addressed by Fisher's fundamental theorem).
\end{subequations}

It is easily checked that
\eqref{eq:partial-change-fftns} plus
\eqref{eq:partial-change-environmental} plus
\eqref{eq:partial-change-genetic-non-fftns} gives
\eqref{eq:total-change}.  Thus we have decomposed the total change
in mean fitness in one generation into three parts, one of which is
the part of the change addressed by Fisher's fundamental theorem and the
others have biological interpretations.

We intend to estimate, discuss, and interpret each of these three differences
separately.  We also need to derive standard errors for each.

\subsection{Other Parts of Change in Mean Fitness}

Other parts of the total change in mean fitness from one generation to the next
are of interest.
\begin{subequations}
\begin{equation} \label{eq:partial-change-foo}
   \sum\nolimits_i \bigl(\hat{q}_i^\text{prev} \hat{\mu}_i^\text{off} -
   p_i^\text{prev} \hat{\mu}_i^\text{par}\bigr)
\end{equation}
is the genetically based change in mean fitness due to natural selection), as
expressed in the environment in which the offspring grow.

\begin{equation} \label{eq:partial-change-bar}
   \sum\nolimits_i {p}_i^\text{prev} \bigl(\hat{\mu}_i^\text{par} -
    \hat{\mu}_i^\text{prev}\bigr)
\end{equation}
is the change in mean fitness due solely to different environments in the
two years.  (This equation is just like
\eqref{eq:partial-change-genetic-non-fftns} except the weights are different:
\eqref{eq:partial-change-bar} uses the (assumed) population weights (before
any of the experiments) and \eqref{eq:partial-change-genetic-non-fftns} 
uses the weights after selection in the first subexperiment, which are the
weights (according to theory) that apply to the offspring in both equations.

\end{subequations}

\section{Standard Errors}

\begin{thebibliography}{}

\bibitem[Geyer, et al.(2022)Geyer, Kulbaba, Sheth, Pain, Eckhart,
    and Shaw]{zenodo}
Geyer, C.~J., Kulbaba, M.~W., Sheth, S.~N., Pain, R.~E., Eckhart, V.~M.,
    and Shaw, R.~G. (2022).
\newblock Correction for Kulbaba et al. (2019).
\newblock \emph{Evolution}, \textbf{76}, 3074.
\newblock \doi{10.1111/evo.14607}.
\newblock Supplementary material, version 2.0.1.
\newblock \doi{10.5281/zenodo.7013098}.

\end{thebibliography}

\end{document}
