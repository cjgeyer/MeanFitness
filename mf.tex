
\documentclass[11pt]{article}

\usepackage{amstext}
\usepackage{indentfirst}

\begin{document}

\title{Estimating Population Mean Fitness and Change Thereof When
    the Population Distribution is Unknown}

\author{Charles J. Geyer \and Ruth G. Shaw}

\maketitle

We have an experiment in which (Darwinian) fitness (measured as lifetime
number of offspring per individual) was measured under wild conditions.
But the individuals in the experiment were not a random sample from
a population but rather pedigreed experimental crosses.  There are no data on
population allele frequencies, and even if there were, there would be
no good way to treat these experimental crosses as members of some
population.  Thus we simply assume some population distribution (of
pedigreed crosses).  All of our inferences will be sensitive to this
assumption, but we can check how sensitive (by trying different assumptions
or doing other mathematical sensitivity analysis).

The particular experiment at hand has five subexperiments laid out as follows.
\begin{center}
\begin{tabular}{ccc}
year one & year two & year three \\
\hline
parents & parents & parents \\
        & offspring & offspring
\end{tabular}
\end{center}
Here ``parents'' means individuals from the pedigreed crosses (seeds from
the same crosses were planted in each year).  For short, we will say these
individuals are from the same ``family'' if they are known full sibs
(have the same parents).
Here ``offspring'' means individuals whose dam was a ``parent'' from the
previous year (sire unknown because of open pollination).
For short, we will say these
individuals are from the same ``family'' as their dam in the previous year.

\pagebreak[3]

If
\begin{itemize}
\item $p_i$ is the (assumed) population frequency of family $i$, and
\item $\hat{\mu}_i$ is (estimated) mean fitness for family $i$,
\end{itemize}
then
\begin{itemize}
\item $\sum_i p_i \hat{\mu}_i$ is (estimated) mean fitness for
that subexperiment, where the sum runs over all families.
\end{itemize}

We also need to know the (estimated) population frequency
\emph{after selection}, which is
$$
   \hat{q}_i = \frac{p_i \hat{\mu}_i}{\sum_j p_j \hat{\mu}_j}
$$
where the sum in the denominator runs over all families.

Now we need to consider three subexperiments
\begin{itemize}
\item a ``parent'' subexperiment in one year (denoted by superscript prev),
\item an ``offspring'' subexperiment in the next year (denoted
    by superscript off), and
\item the ``parent'' subexperiment in the same year as the ``offspring
    experiment (denoted by superscript par).
\end{itemize}

The total change in mean fitness in one generation is
$$
   \textstyle \sum_i \bigl[
   \hat{q}_i^\text{prev} \hat{\mu}_i^\text{off}
   -
   p_i^\text{prev} \hat{\mu}_i^\text{prev}
   \bigr]
$$
we break this up as the sum of three differences in mean fitness
\begin{itemize}
\item $\sum_i (\hat{q}_i^\text{prev} - p_i^\text{prev}) \hat{\mu}_i^\text{prev}$
    is the change in mean fitness that Fisher's fundamental theorem of
    natural selection tries to address (change in fitness due to natural
    (viability) selection),
\item $\sum_i \hat{q}_i^\text{prev} (\hat{\mu}_i^\text{par} -
    \hat{\mu}_i^\text{prev})$
    is the change in mean fitness due to different environments in the
    two years, and
\item $\sum_i \hat{q}_i^\text{prev} (\hat{\mu}_i^\text{off} -
    \hat{\mu}_i^\text{par})$
    is the change in mean fitness due to differences in genetics of the
    parent and offspring subjects not due to viability selection
    (something not addressed by Fisher's fundamental theorem).
\end{itemize}
We intend to estimate, discuss, and interpret each of these three differences
separately.  We also need to derive standard errors for each.

\end{document}
